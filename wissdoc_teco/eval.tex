%% eval.tex
%% $Id: eval.tex 5 2005-10-10 20:55:48Z bless $

\chapter{Evaluierung}
\label{ch:Evaluierung}
%% ==============================
Hier kommt der Nachweis, dass das in Kapitel~\ref{ch:Entwurf}
entworfene Konzept auch funktioniert. Leistungsmessungen einer
Implementierung werden auch immer gerne gesehen.

Bla fasel\ldots

%% ==============================
\section{Probandinnen}
%% ==============================
\label{ch:Evaluierung:sec:Abschnitt1}

Die Studie wurde begonnen mit 25 Probandinnen.
Alle Probandinnen waren Studentinnen und zwischen 20 und 26 Jahren alt.
Davon studieren 20 Probandinnen Informatik. 
Von den 25 Probandinnen waren 6 weiblich und 19 männlich.
Die Smartphones von drei Probandinnen verfügten über Android Marshmallow, während die 22 verbleibenden Probandinnen über Android Lollipop verfügten.  
Aufgrund von äußeren Umständen sind zwei Probandinnen von der Studie zurücktreten:
Bei einer Probandin gab es einen, von der Studie unabhängigen, kritischen Systemfehler auf ihrem Smartphone, was dazu führte, dass dieses unbenutzbar wurde sie die Studie nicht fortgeführen konnte.
%und sie auf ihr Ersatzsmartphone wechseln musste.
%Dieses ist ein Windows Phone und dementsprechend konnte die Studie dort nicht fortgeführt werden.
Die andere bemerkte zu spät, dass sie nur in einem zu kleinen Zeitfenster and der Studie teilnehmen könnte und trat nach dem Ausfüllen des NEO-PI-R Fragebogens zurück.


%% ==============================
\section{Ablauf}
%% ==============================
\label{ch:Entwurf:sec:Abschnitt2}

Nachdem die Vorstudie, wie im vorangegangenen Kapitel beschrieben, abgeschlossen ist, und potenzielle kleine Änderungen an der Applikation durchgeführt wurden, beginnt die Hauptstudie.
In dieser werden zunächst die Testprobanden ans Teco eingeladen, um dort den NEO-PI-R Fragebogen auszufüllen.
Dazu werden circa drei bis vier Termine innerhalb einer Kalenderwoche angeboten, aus denen die Testprobanden wählen können.
Dies geschieht aus mehreren Gründen. 
Erstens verringert dies, das Risiko, das manche Studienteilnehmerinnen aufgrund von terminlichen Konflikten nicht teilnehmen können.
Zweitens sollen die Daten über einen möglichst ähnlichen Zeitraum gesammelt werden.
Zuletzt ist es so, dass durch das Aufteilen der Probandengruppe eine persönlichere Atmosphäre beim Auftakt der Studie möglich ist
und potenzielle Fragen im Zwiegespräch besser geklärt werden können.
\par
Zu Beginn der Einführung wird den Probandinnen zunächst das Wesen, Bedeutung und potenzielle Tragweite der Studie erläutert.
Potenzielle Risiken und Nutzen werden aufgeführt.
Hier gibt es die Möglichkeit Fragen an die Studienleiterin zu stellen.
Nachdem dies geschehen ist, werden die Probandinnen gebeten, sofern sie nach den Erläuterungen keine Fragen mehr haben und mit der Teilnahme einverstanden sind, eine Einverständniserklärung zu unterschreiben.
Ein zweites Exemplar ist für die Probandin selbst, als von der Studienleiterin unterschriebenen Zusicherung, dass der Rücktritt von den Studie ohne Angabe von Gründen und ohne Nachteile für die Probandin zu jedem Zeitpunkt möglich ist.
Sobald dies abgehandelt ist, können die Probandinnen beginnen den NEO-PI-R Persönlichkeitstest auszufüllen.
Bevor dies geschieht, sollen die Probanden ein selbstgewähltes Pseudonym auf die zweite Seite des Tests schreiben, sodass die gesammelten Daten zu den Tests zugeordnet werden kann, ohne die Anonymität der Probandinnen einzuschränken.
Das alles wird je nach Bearbeitungsgeschwindigkeit ungefähr 35 bis 45 Minuten einnehmen.
\par
Nachdem diese verstrichen sind, und alle Probandinnen ihren Test abgeschlossen haben, werden auch diese eingesammelt.
Nun kann die Applikation auf den mitgebrachten Smartphones der Probandinnen installiert werden.
Im Vorhinein ist die APK der Applikation so in der Cloud hinterlegt worden, dass sie per Scannen eines QR Codes (siehe Abbildung \ref{fig:qrcode}) heruntergeladen werden kann.
Die Installation der Applikation von der APK läuft ab, wie jede andere Applikationsinstallation auf einem Android Gerät auch.
Potenziell muss hier jedoch kurzzeitig die Installation von Applikationen aus nicht verifizierten Quellen freigeschaltet werden, sollte dies nicht sowieso schon der Fall sein.
Nach dem Bestätigen der Basis Berechtigungen ist die Applikation installiert.
Um sie betriebsbereit zu machen müssen die Nutzerinnen die erweiterten Berechtigungen von Hand erteilen.
Dadurch sollte die Applikation betriebsbereit sein.
\par
Um die ordnungsgemäße Funktionstüchtigkeit zu gewährleisten sind in der Applikation Tests implementiert, die nun durchgeführt werden sollen.
Sind diese erfolgreich durchgeführt worden, so beginnt der Datensammelungszeitraum der Studie.
Dieser endet zwischen 10 und 20 Tagen später.
Nachdem die von der Applikation gesammelten Daten exportiert und zurück bei der Studienleiterin angekommen sind, 
ist die Studie beendet.

\begin{figure}[h]
    \centering
    \includegraphics{images/qrcode.png}
    \caption{QR Code}
    \label{fig:qrcode}
\end{figure}


%% ==============================
\section{Probleme während dem Studienablauf}
%% ==============================

Gleich zu Beginn der Studie, bei der Installation der Applikation gab es bei einer Probandin Probleme mit der Android Version:
Der Optionspunkt in der einer Applikation der Zugang zu den UsageStat Daten ermöglicht, wurde vom Hersteller auf dem LG Gerät zumindest in der Default ROM ersatzlos entfernt.
Dies führt dazu, dass die Datensammelapplikation der Studie nicht auf alle Daten Zugriff bekommt, auf die sie Zugriff benötigt.
Die Probandin, selbst eine Android Programmiererin, die mehrere aktuelle Android Smartphones besitzt, bot daraufhin an für die Dauer der Studie auf ein anderes Gerät umzusteigen.
Ein einfaches Überführen der SIM Karte in das Zeitgerät umgeht das Problem.
\par
Im Verlauf der Studie beobachteten mehrere Probandinnen,  dass die von der App erhaltenen Werte für die Vordergrundzeit verschiedener Applikationen ungenau oder zu gering seien.
Dies scheint unabhängig von der sammelnden Applikation zu sein, sondern ein unerwartetes Verhalten des Android Betriebssystems selbst, dass falsche Daten aus der Schnittstelle ausgibt.
Aufgrund der Unzugänglichkeit der Problematik und Zeitbedenken wurden, nach Rücksprache mit der betreuenden Mitarbeiterin, die gesammelten UsageStat Daten für unzuverlässig erklärt.
Dementsprechend werden diese Daten in der Evaluation nicht mehr berücksichtigt.
\par
Nach Ablauf der geplanten Studienzeit bekamen alle Studienteilnehmer die Aufforderung die gesammelten Daten zu exportieren und der Studienleiterin zukommen zu lassen. 
In 20 von 23 Fällen funktionierte dies perfekt.
Jedoch gab es einen Fall, in dem das bereits aufgefallene Problem \cite{androidbug} mit der Android Dateiübertragung auftrat.
Dies konnte die Probandin nach einem Hinweis durch einen Neustart des Geräts beheben.
Die anderen beiden Fälle wurden durch zerstreute Probandinnen und das System der Anonymisierung beziehungsweise Pseudonymisierung hervorgerufen.
Da der Studienleiterin nur die Pseudonyme der Probandinnen bekannt waren, gestaltete es sich unmöglich die beiden fehlenden Probandinnen direkt auf das Fehlen ihrer Datensätze hinzuweisen.
Mit sieben Tagen Verspätung und nach etlichen Nachrichten an verschiedene Teilnehmerinnen waren die gesammelten Daten dann vollständig.

\section{Feedback}

Nach Abschluss der Studie wurden die Probandinnen dazu aufgefordert einen kurzen Feedback Fragebogen auszufüllen.
Die gestellten Fragen "`Hat sich die Nutzung Ihres Smartphones während der Studie/durch die App verändert? Wie?"', "`Welche Kommunikations- oder Social Media-Apps verwendest du noch, welche nicht mitgeloggt wurden?"' und "`Freies Feedback: Gibt es noch etwas, dass ihr mir sagen möchtet?"' waren alle drei als Freitext Fragen gestellt.
\par
Die Antworten bezüglich der ersten Frage waren ausnahmslos verneinend.
Der Grund hierfür ist vermutlich bei der Abwesenheit von Interaktion mit der Applikation zu finden.
Die Abwesenheit von Interaktion wurde sogar beim freien Feedback einmal explizit als positiv hervorgehoben und die Applikation wurde als "`nicht bemerkbar"' beschrieben.
\par
Antworten auf die zweite Frage waren dann schon weniger homogen:
Mehrere Probandinnen wiesen hier auf "`Conversations"' hin. 
Das ist ein Android Jabber / XMPP Client, mit dem verschiedene Dienste, die das XMPP Übertragungsformat nutzen, in einer App bündeln kann.
Viele Einstellungsmöglichkeiten, wie zum Beispiel die Verschlüsselung über OTR oder OpenPGP, machen diese Applikation für den Normalnutzer eher unhandlich, für Informatikstudentinnen aber umso attraktiver.
Angesichts der am Ende doch sehr Informatik lastigen Probandinnen wäre die Aufnahme dieser App in die Liste der zu betrachtenden Apps durchaus zu rechtfertigen gewesen.
Einige obskure Twitterclients wurden noch angeführt.
Aufgrund der schieren Masse an verschiedenen Twitter Applikationen ist es schwerlich möglich diese alle von Entwicklerseite abzudecken. 
Im Nachhinein wäre eine denkbare Lösung für dieses Problem, dass der Nutzer selbst eintragen kann, wie das Package des Twitter Clients seiner Wahl heißt.
Da das Ändern von Resource Daten durch Userinput nicht möglich ist, müsste man zwar das Konzept ändern mit dem die relevanten Packagenamen gespeichert werden, doch dies scheint ein lösbares Problem zu sein.
\par
Eine andere Antwort umfasst sowohl die Frage nach nicht mitgeloggten Applikationen, als auch das freie Feedback Feld:
Eine der teilnehmenden Probandinnen ist in Asien geboren und hat noch immer regen kontakt mit Freunden beziehungsweise Familie von dort, 
die selbstverständlich nicht die in Deutschland, beziehungsweise der westlichen Welt herkömmlich benutzten Applikationen benutzen, sondern die Asiatischen.
Beispiele die angeführt wurden waren QQ, WeChat oder Ren Ren.
Demensprechend bemerkte sie, dass insgesamt die Auswahl der betrachteten Applikationen sehr "`westlich"' war.
Während dies unzweifelhalt wahr ist, so ist der Anspruch über Deutschland hinaus solche Muster zu untersuchen sicher etwas außerhalb des Rahmens einer Bachelorarbeit.
Für weitergehende Forschungen ist jedoch sicher viel Platz auf diesem Gebiet, unter anderem vielleicht dem Unterschied zwischen Applikationsverhalten in Europa, Nord- beziehungsweise Süd-Amerika und Asien. 
\par

TODO: Freies Feedback, e.g. DER FRAGEBOGEN

\section{Auswertung der Daten}

Das Umsetzen der erhaltenen Daten in eine aussagekräftige Übersicht war für die gesammelten Call und Message Log Daten nicht notwendig, da sie bereits in einem solchen exportiert werden, für die Liste an Notifications ist dies jedoch dringend notwendig.
Dafür wurde eine Java Command Line Applikation geschrieben, die Daten aus verschiedenen Dateien einließt und am Ende die aggregierten Daten in eine Datei schreibt.
\par
Es wird ein Pfad zu dem Ordner in dem sich die zu betrachtenden Dateien befinden angegeben.
Über alle in sich diesem Ordner befindlichen Datein wird iteriert und für jede wird zunächst der gesamte Inhalt der Datei mit der Methode \emph{readStringFromFile()} in einen String geschrieben.
Mit der \emph{split()} Methode und dem regulären Ausdruck  $(\backslash r?\backslash n)|,|(\backslash t)$ wird der String in die Comma Separated Values aufgeteilt.
Da die ursprüngliche Tabelle 5 Spalten hatte sind nun die Attribute der i. Notification im Array in den Positionen $ [i * 5 + 0]$ bis $ [i * 5 + 4]$ gespeichert.
Diese werden nun in Objekte \emph{Notification} mit den entsprechenden Attributen und einem Enum für die absendende Applikation geparst.
Alle so erstellten Objekte werden dann in eine ArrayList geschrieben auf der weitergearbeitet wird.
Zur einfacheren Werteberechnung für einzelne Applikationen wird nun für jede mögliche Applikation eine eigene ArrayListe angelegt, in die nur die Notifications hinzugefügt werden die die entsprechende Applikation als Absender haben.
Für jede dieser ArrayListen werden nun folgende Werte berechnet:

\begin{itemize}
  \item Anzahl Notifications (Länge der ArrayList)
  \item Anzahl Konversationen (Anzahl einzigartiger Titelhashes)
  \item Größte Konversation (Maximum TextLength)
  \item Durchschnittliche Konversationsgröße (Summe TextLength durch Anzahl Konversationen)
  \item Durchschnittliche Nachrichtenlänge (Summe TextLength durch Anzahl Notifications)
\end{itemize}

Diese Berechnungen werden jeweils in einer Methode, ähnlich zum Beispiel \ref{maxconv} durchgeführt, die dann auf auf alle ArrayLists aufgerufen wird.


\begin{lstlisting}[frame=single, caption =  calculateMaxConversationSize(), label=maxconv] 
    public static int calculateMaxConversationSize(ArrayList<Notification> list) {
        HashMap<String, Integer> hmap = new HashMap<>();
        for (Notification noti : list) {
            if (!hmap.containsKey(noti.getTitle())) {
                hmap.put(noti.getTitle(), 1);
            }
            else {
                hmap.put(noti.getTitle(), hmap.get(noti.getTitle())+ 1);
            }
        }
        if (hmap.size() == 0){
            return 0;
        }
        else {
            return (Collections.max(hmap.values()));
        }

    }
\end{lstlisting}


Über alle Daten werden zusätzlich noch folgende Werte berechnet:

\begin{itemize}
  \item Anzahl Applikationen genutzt (Anzahl ArrayLists mit Size > 0)
  \item Anzahl Notifications insgesamt
  \item Anzahl Notifications pro Tag
  \item Durchschnittliche Nachrichtenlänge gesamt
  \item Meist genutzte Applikation
  \item Applikation mit den meisten Konversationen
  \item Applikation mit den durschnittlich längsten Nachrichten  
\end{itemize}

Die so aggregierten Daten werden dann in eine Datei geschrieben von der sie dann an relevante Stellen übertragen werden können.

\section{Ergebnisse}

Die Auswertung der NEO-PI-R Fragebögen ergab, dass von den 23 Probanden, die letztendlich betrachtet wurden, elf eine durchschnittliche, acht eine sehr niedrige, drei eine hohe und nur einer eine niedrige Geselligkeit hatten.
Auf der Skala von 0 bis 4 lag der Durchschnitt bei ungefähr $1.6$ mit einer Standardabweichung von $0.9$ und zwischen dem Minimum von $0.125$ und dem Maximum von $2.75$.
Im Vergleich dazu Werte der Extraversion:
Im Durchschnitt $1.88$ mit Standardabweichung von $0.57$, dem Minimum bei $0.56$ und dem Maximum bei $2.6$.
Die Verteilung der Ergebnisse ist in Diagramm \ref{fig:neoergebnisse} veranschaulicht.

\begin{figure}[h]
    \centering
    \includegraphics{images/NeoErgebnisse.pdf}
    \caption{Verteilung der Geselligkeit und Extraversion der Probandinnen}
    \label{fig:neoergebnisse}
\end{figure}

Durchschnittlich führten die Probandinnen $0.66$ Telefongespräche pro Tag bei einer Standartabweichung von $0.89$.
Drei Probandinnen führten überhaupt keine Telefongespräche und die Probandin mit der höchsten Telefonnutzung tätigte $4.03$ Anrufe pro Tag.
Das ist eine höhere Nutzung als die von SMS, was im Durchschnitt nur $0.38$ Nachrichten pro Tag bei einer Standardabweichung von $0.6$ war.
SMS wurde von vier Probandinnen überhaupt nicht genutzt und das Maximum lag bei $1.86$ Nachrichten pro Tag.
Die Verteilung von Anrufen und Nachrichten ist in Diagramm \ref{fig:callmessagesdistr} dargestellt.

\begin{figure}[h]
    \centering
    \includegraphics{images/callsMessagesDistr.pdf}
    \caption{Verteilung der Anrufe und Nachrichten der Probandinnen}
    \label{fig:callmessagesdistr}
\end{figure}


Bei den gesammelten Notifications wurden durchschnittlich $157.29$ Notifications pro Tag bei einer Standardabweichung von $378.60$ empfangen.
Dies ist einem extremen Ausreißer geschuldet, der über 17 Tage 31716 Notifications erhalten hat. 
Das sind fast $8.5$ mal mehr als der zweithöchste Wert, mit 3755 Notifications.
Entfernt man diesen Wert aus dem Datenpool reduziert sich der Durchschnitt auf $79.63$ Notifications pro Tag mit einer Standartabweichung von $69.79$
\par
Die Verteilung der meistgenutzten Applikationen jeder Teilnehmerin sind in Diagramm \ref{fig:mostusedapp} dargestellt.
\par
TODO: Beschreibung korrelationen

\begin{figure}[h]
    \centering
    \includegraphics{images/MostUsedApp.pdf}
    \caption{Verteilung der meistgenutzten Applikationen}
    \label{fig:mostusedapp}
\end{figure}



%% ==============================
\section{Zusammenfassung}
%% ==============================
\label{ch:Evaluierung:sec:zusammenfassung}

Am Ende sollten ggf. die wichtigsten Ergebnisse nochmal in \emph{einem}
kurzen Absatz zusammengefasst werden.

%%% Local Variables: 
%%% mode: latex
%%% TeX-master: "diplarb"
%%% End: 
