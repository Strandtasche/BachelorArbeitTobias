%% eval.tex
%% $Id: eval.tex 5 2005-10-10 20:55:48Z bless $

\chapter{Evaluierung}
\label{ch:Evaluierung}
%% ==============================
Hier kommt der Nachweis, dass das in Kapitel~\ref{ch:Entwurf}
entworfene Konzept auch funktioniert. Leistungsmessungen einer
Implementierung werden auch immer gerne gesehen.

Bla fasel\ldots

%% ==============================
\section{Probandinnen}
%% ==============================
\label{ch:Evaluierung:sec:Abschnitt1}

Die Studie wurde begonnen mit 25 Probandinnen.
Alle Probandinnen waren Studenten und zwischen 20 und 26 Jahren alt.
Davon studerien 20 Probandinnen Informatik. 
Von den 25 Probandinnen waren 6 weiblich und 19 männlich.
Die Smartphones von Drei Probandinnen verfügten über Android Marshmallow während die 22 verbleibenden Probandinnen über Android Lollipop verfügten.  
Aufgrund von äußeren Umständen sind zwei Probandinnen von der Studie zurück treten.


%% ==============================
\section{Abschnitt 2}
%% ==============================
\label{ch:Evaluierung:sec:Abschnitt2}

Bla fasel\ldots

%% ==============================
\section{Zusammenfassung}
%% ==============================
\label{ch:Evaluierung:sec:zusammenfassung}

Am Ende sollten ggf. die wichtigsten Ergebnisse nochmal in \emph{einem}
kurzen Absatz zusammengefasst werden.

%%% Local Variables: 
%%% mode: latex
%%% TeX-master: "diplarb"
%%% End: 
