%% eval.tex
%% $Id: eval.tex 5 2005-10-10 20:55:48Z bless $

\chapter{Evaluierung}
\label{ch:Evaluierung}
%% ==============================
Hier kommt der Nachweis, dass das in Kapitel~\ref{ch:Entwurf}
entworfene Konzept auch funktioniert. Leistungsmessungen einer
Implementierung werden auch immer gerne gesehen.

Bla fasel\ldots

%% ==============================
\section{Probandinnen}
%% ==============================
\label{ch:Evaluierung:sec:Abschnitt1}

Die Studie wurde begonnen mit 25 Probandinnen.
Alle Probandinnen waren Studentinnen und zwischen 20 und 26 Jahren alt.
Davon studerien 20 Probandinnen Informatik. 
Von den 25 Probandinnen waren 6 weiblich und 19 männlich.
Die Smartphones von Drei Probandinnen verfügten über Android Marshmallow während die 22 verbleibenden Probandinnen über Android Lollipop verfügten.  
Aufgrund von äußeren Umständen sind zwei Probandinnen von der Studie zurücktreten:
Bei einer Probandin gab es einen von der Studie unabhängen kritischen Systemfehler auf ihrem Smartphone was dazu führte, dass dieses unbenutzbar wurde und sie auf ihr Ersatzsmartphone wechseln musste.
Dieses ist ein Windows Phone und dementsprechend konnte die Studie dort nicht fortgeführt werden.
Die andere bemerkte zu spät, dass sie nur in einem zu kleinen Zeitfenster and der Studie teilnehmen könnte und trat nach dem ausfüllen des NEO-PI-R Fragebogens zurück.


%% ==============================
\section{Ablauf}
%% ==============================
\label{ch:Entwurf:sec:Abschnitt2}

Nachdem die Vorstudie, wie im vorangegangenen Kapitel beschrieben, abgeschlossen ist, und potenzielle kleine Änderungen an der Applikation durchgeführt wurden beginnt die Hauptstudie.
In dieser werden zunächst die Testprobanden ans Teco eingeladen um dort den NEO-PI-R Fragebogen auszufüllen.
Dazu werden circa drei bis vier Termine innerhalb einer Kalenderwoche angeboten, aus denen die Testprobanden wählen können.
Dies geschieht aus mehreren Gründen. 
Erstens verringert dies, das Risiko, das manche Studienteilnehmerinnen aufgrund von terminlichen Konflikten nicht teilnehmen können.
Zweitens sollen die Daten über einen möglichst ähnlichen Zeitraum gesammelt werden.
Zuletzt ist es so, dass durch das Aufteilen der Probandengruppe ist eine persönlichere Atmosphäre beim Auftakt der Studie möglich
und potenzielle Fragen können im Zwigespräch besser geklärt werden als wenn alle Probandinnen gleichzeitig in die Studie einweist.
\par
Zu Beginn der Einführung wird den Probandinnen zunächst das Wesen, Bedeutung und potenzielle Tragweite der Studie erläutert.
Potenzielle Risiken und Nutzen werden aufgeführt.
Hier gibt es die Möglichkeit Fragen an den Studienleiter zu stellen.
Nachdem dies geschehen ist, werden die Probandinnen gebeten, sofern sie nach den Erläuterungen keine Fragen mehr haben und mit der Teilnahme einverstanden sind, eine Einverständniserklärung zu unterschreiben.
Die andere ist für die Probandin selbst, als von der Studienleiterin unterschriebenen Zusicherung, dass der Rücktritt von den Studie ohne Angabe von Gründen und ohne Nachteile für die Probandin zu jedem Zeitpunkt möglich ist.
Sobald dies abgehandelt ist können die Probandinnen beginnen den NEO-PI-R Persönlichkeitstest auszufüllen.
Bevor dies geschieht sollen die Probanden ein selbstgewähltes Pseudonym auf die zweite Seite des TeDsts schreiben, sodass die gesammelten Daten zu den Tests zugeordnet werden kann ohne die Anonymität der Probandinnen einzuschränken.
Das alles wird je nach Bearbeitungsgeschwindigkeit ungefähr 35 bis 45 Minuten einnehmen.
\par
Nachdem diese verstrichen sind, und alle Probandinnen ihren Test abgeschlossen haben, werden auch diese eingesammelt.
Nun kann die Applikation auf den mitgebrachten Smartphones der Probandinnen installiert werden.
Im Vorhinein ist die APK der Applikation so in der Cloud hinterlegt worden, dass sie per scannen eines QR Codes (siehe Abbildung \ref{fig:qrcode}) heruntergeladen werden kann.
Die Installation der Applikation von der APK läuft ab, wie jede andere Applikationsinstallation auf einem Android Gerät auch.
Potenziell müssen hier jedoch kurzzeitig die Installation von Applikationen aus nicht verifizierten Quellen freigeschaltet werden, sollte dies nicht sowieso schon der Fall sein.
Nach dem Bestätigen der Basis Berechtigungen ist die Applikation installiert.
Um sie betriebbereit zu machen müssen die Nutzerinnen die erweiterten Berechtigungen von Hand erteilen.
Nun sollte die Applikation betriebsbereit sein.
\par
Um die ordnungsgemäße Funktionstüchtigkeit zu gewährleisten sind in der Applikation Tests implementiert, die nun durchgeführt werden sollen.
Sind diese erfolgreich durchgeführt worden, so beginnt der Datensammelungszeitraum der Studie.
Dieser endet zwischen 10 und 20 Tagen später.
Nachdem die von der Applikation gesammelten Daten exportiert und zurück bei der Studienleiterin angekommen sind, 
ist die Studie beendet.

\begin{figure}[h]
    \centering
    \includegraphics{images/qrcode.png}
    \caption{QR Code}
    \label{fig:qrcode}
\end{figure}


%% ==============================
\section{Probleme während dem Studienablauf}
%% ==============================

Gleich zu Beginn der Studie, bei der Installation der Applikation gab es bei einer Probandin Probleme mit der Android Version:
Der Optionspunkt in der einer Applikation der Zugang zu den UsageStat Daten wurde vom Hersteller auf dem LG Gerät zumindest in der Default ROM ersatzlos entfernt.
Dies führt dazu, dass die Datensammelapplikation der Studie nicht auf alle Daten Zugriff bekommt, auf die sie Zugriff benötigt.
Glücklicherweise besitzt die Probandin, selbst eine Android Programmiererin, mehrere aktuelle Android Smartphones, sodass das einfach übertragen der SIM Karte in ein anderes Gerät das Problem umgeht.
\par
Während dem Verlauf der Studie beobachteten mehrere Probandinnen,  dass die von der App erhaltenen Werte für die Vordergrundzeit verschiedener Applikationen ungenau oder zu gering seien.
Dies scheint ein Unabhängig von der sammelnden Applikation zu sein, sondern ein unerwartetes Verhalten des Android Betriebssystems selbst, das die falschen Daten aus der Schnittstelle ausgibt.
Aufgrund der Unzugänglichkeit der Problematik und Zeitbedenken wurde, nach Rücksprache mit der betreuenden Mitarbeiterin, die gesammelten UsageStat Daten für unzuverlässig erklärt.
Dementsprechend werden diese Daten in der Evaluation nicht mehr berücksichtigt.
\par
Nach Auflauf der geplanten Studienzeit bekamen alle Studienteilnehmer die Aufforderung die gesammelten Daten zu exportieren und der Studienleiterin zukommen zu lassen. 
In 20 von 23 Fällen funktionierte dies Perfekt.
Jedoch gab es einen Fall in dem das bereits aufgefallene Problem \cite{androidbug} mit der Android Dateiübertragung auftrat.
Dies konnte die Probandin nach einem Hinweis durch einen Neustart des Geräts behoben werden.
Die anderen beiden Fälle wurden durch zerstreute Probandinnen und das System der Annonymisierung beziehungsweise Pseudonymisierung hervorgerufen.
Da der Studienleiterin nur die Pseudonyme der Probandinnen bekannt waren, gestaltete es sich unmöglich die beiden fehlenden Probandinnen direkt auf das Fehlen ihrer Datensätze hinzuweisen.
Mit sieben Tagen Verspätung und nach etlichen Nachrichten an verschiedene Teilnehmerinnen waren die gesammelten Daten dann vollständig.


\section{Ergebnisse}

TODO: 


%% ==============================
\section{Zusammenfassung}
%% ==============================
\label{ch:Evaluierung:sec:zusammenfassung}

Am Ende sollten ggf. die wichtigsten Ergebnisse nochmal in \emph{einem}
kurzen Absatz zusammengefasst werden.

%%% Local Variables: 
%%% mode: latex
%%% TeX-master: "diplarb"
%%% End: 
