%% zusammenf.tex
%% $Id: zusammenf.tex 4 2005-10-10 20:51:21Z bless $
%%

\chapter*{Abstract}
\label{ch:Abstract}
%% ==============================

A person’s gregariousness is a key factor in the way they interact with both their peers and their environment. 
Observing behaviour in mobile communication has become increasingly easier with the rise of the smartphone as an ever- present companion and as a tool for communication. 
However, the majority of work in that field does not consider the changes in communication that come along with the broad acceptance of messenger applications.
\par
This thesis presents a study in analysis, design and execution. 
It explores the correlation between gregariousness as an aspect of extraversion, and communication data collected on a user's smartphone. 
For this purpose, both data from conventional communication methods, such as calls or text messages, and APP are collected and examined for correlation with the users’ gregariousness. 
The data is collected by a specifically developed application on the users’ smartphone which treats incoming notifications as valid representations of relevant messages.
\par
For this study, 23 users both filled out the 240 questions NEO-PI-R questionnaire and ran the application in the background of their phones for a period of ten to 20 days. The resulting data was merged and compared after the conclusion of the study. 
Even though a lot of good data was acquired by the application, the study did not manage to find a significant correlation between gregariousness and the usage of social media and instant messaging. 
However, this does not mean that this field of study is hereby completed but rather that possibilities for further research need to be explored.






%%% Local Variables: 
%%% mode: latex
%%% TeX-master: "diplarb"
%%% End: 
