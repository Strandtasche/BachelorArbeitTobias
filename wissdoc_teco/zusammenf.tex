%% zusammenf.tex
%% $Id: zusammenf.tex 4 2005-10-10 20:51:21Z bless $
%%

\chapter{Zusammenfassung und Ausblick}
\label{ch:Zusammenfassung}
%% ==============================


In dieser Arbeit wurde der Zusammenhang zwischen der Geselligkeits- beziehungsweise Extraversions-Facette
einer Persönlichkeit und der Smartphone Nutzung im Kontext von Social Media und Kommunikation untersucht.
Es wurden verwandte Arbeiten vorgestellt und bewertet. 
Diese Arbeiten nutzten verschieden geeignete Methoden um die Persönlichkeit ihrer Probandinnen zu quantifizieren und ihre Nutzung von Smartphones zu festzuhalten.
Der Einsatz von Persönlichkeitsfragebögen aus der NEO Familie ist geeignet um die Ausprägung der 5 Dimensionen des Big Five Modells zu messen.
Vom Smartphone selbst gesammelte Daten sind denen aus Befragungen der Probandin vorzuziehen.
Als Bachelorarbeit existiert eine Grenze wie viel Zeit und Aufwand in eine Studie investiert werden kann. 
\par
Googles Androidbetriebssystem stellt eine Reihe an Schnittstellen und Möglichkeiten zur Verfügung, wie eine mit den entsprechenden Berechtigungen ausgestattete Applikation
Daten bezüglich der Smartphone Nutzung sammeln kann.
Es wurden nach Abwägungen bezüglich Zugänglichkeit, Aussagekraft und Schutz der Privatsphäre Notifications und UsageStat als Repräsentation für die Interaktion mit Applikationen gewählt.
Ergänzt werden diese von aggregierten Daten aus Anrufen und SMS als Vertreter der konventionellen Nutzung von Mobiltelefonen.
\par
Daraufhin wurde eine Studie entworfen und nach der Durchführung einer kleineren Vorstudie durchgeführt, die anhand 25 Testprobandinnen Daten, aus einem durchgeführten Selbstbeurteilungsfragebogen und
 mittels einer dazu entwickelten Android Applikation, bezüglich dieses Zusammenhangs sammelt.

Folgend darauf wurde das Durchführen der Studie sowie die aufgekommenen Problemen beschrieben und durchgeführte Lösungsansätze diskutiert.
Vor Beginn des Datensammelns füllten die 25 Probandinnen die 240 Fragen des NEO-PI-R Fragebogens aus, aus denen dann jeweils ein numerischer Wert für die Extraversion und die Geselligkeit berechnet wurde.
Danach ließen sie für zwischen 12 und 28 Tagen die Applikation im Hintergrund laufen, sodass während dieser Zeit Daten aufzeichnet werden könnten.

\par
In den so gesammelten Daten konnte keine Korrelation zwischen der Nutzung von Applikationen und der Geselligkeit der Nutzerin festgestellt werden. 
Die einzige findbare Korrelation war bei den konventionellen Mobiltelefonfeatures zu finden.
Zwischen der Länge der Telefonanrufe und der Geselligkeit zeigte sich ein signifikanter ($p < 0.05$) Zusammenhang.
\par
Dass im Rahmen dieser Arbeit kein signifikanter Zusammenhang zwischen Social Media- beziehungsweise Instantmessaging-Nutzung gefunden werden konnte bedeutet aber keinesfalls,
dass diese Richtung der Forschung nicht weiter getrieben werden sollte.
Die hier durchgeführte Studie litt unter den Limitationen, die ihr auferlegt waren und dem wegfallen der UsageStat Daten.
Eine Studie mit einer größeren, heterogeneren und ausgeglicherenen Probandengruppe, die über einen längere Laufzeit verfügt könnte einen tieferen Einblick in das Thema geben.
Als zusätzlich zu erhebende Daten wäre es sehr interessant tatsächlich ausgehende Nachrichten zu betrachten.
Alternativ ist auch eine Erweiterung der Fragestellung auf andere Unterfacetten der Extraversion oder auf andere Dimensionen des Big Five erweitern um andere Aspekte der Kommunikation mit dem Smartphone zu betrachten.
\par
Abschließend kann man sagen, dass die große Menge an personenbezogenen Daten,
die jeden Tag durch das persönliche Smartphone fließt, dieses zu einem sehr attraktiven Werkzeug zur Betrachtung der Persönlichkeit seiner Benutzerin macht.
Dies begrenzt sich nicht nur auf die Geselligkeit oder die Extraversion: Die Aspekte die nicht so betrachtet werden können sind verschwindend wenige.
Dies ist ein interessantes Forschungsthema, mit viel Potenzial in der Zukunft, jedoch muss Acht gegeben werden.
Diese Forschung kann leicht den Vorteil der Nutzerin aus den Augen verlieren und missbraucht werden. 


%%% Local Variables: 
%%% mode: latex
%%% TeX-master: "diplarb"
%%% End: 
