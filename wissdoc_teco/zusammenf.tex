%% zusammenf.tex
%% $Id: zusammenf.tex 4 2005-10-10 20:51:21Z bless $
%%

\chapter{Zusammenfassung und Ausblick}
\label{ch:Zusammenfassung}
%% ==============================


In dieser Arbeit wurde der Zusammenhang zwischen der Geselligkeits- beziehungsweise Extraversions-Facette
einer Persönlichkeit und der Smartphone Nutzung im Kontext von Social Media und Kommunikation untersucht.
Es wurden verwandte Arbeiten vorgestellt und bewertet.
\par
TODO: FUELLEN MIT RELATED WORK
\par
Daraufhin wurde eine Studie entworfen und nach der Durchführung einer kleineren formativen Studie durchgeführt, die anhand 25 Testprobandinnen Daten, aus einem durchgeführten Selbstbeurteilungsfragebogen und
 mittels einer dazu entwickelten Android Applikation, bezüglich dieses Zusammenhangs sammelt.

Folgend darauf wurde das Durchführen der Studie sowie die aufgekommenen Problemen beschrieben und durchgeführte Lösungsansätze diskutiert.
\par
TODO: BESCHREIBUNG STUDIE
\par
Die so gesammelten Daten konnten belegen, dass *INSERT THESE HIER*.\\
*so viele Datendinge hier*\\
*ganz viele*
\par
TODO: AUSBLICK AUF DIE ZUKUNFT!
\par
Abschließend kann man sagen, dass die große Menge an personenbezogenen Daten,
die jeden Tag durch das persönliche Smartphone fließt, dieses zu einem sehr attraktiven Tool zur Betrachtung der Persönlichkeit seiner Benutzerin macht.
Dies begrenzt sich nicht nur auf die Geselligkeit oder die Extraversion: Die Aspekte die nicht so betrachtet werden können sind verschwindend wenige.
Dies ist ein interessantes Forschungsthema, mit viel Potenzial in der Zukunft, jedoch muss Acht gegeben werden.
Diese Forschung kann leicht den Vorteil der Nutzerin aus den Augen verlieren und missbraucht werden. 


%%% Local Variables: 
%%% mode: latex
%%% TeX-master: "diplarb"
%%% End: 
