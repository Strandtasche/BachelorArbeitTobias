%% grundlagen.tex
%% $Id: grundlagen.tex 28 2007-01-18 16:31:32Z bless $
%%

\chapter{Grundlagen}
\label{ch:Grundlagen}
%% ==============================
Die Grundlagen müssen soweit beschrieben
werden, dass ein Leser das Problem und
die Problemlösung  versteht.Um nicht zuviel 
zu beschreiben, kann man das auch erst gegen 
Ende der Arbeit schreiben.

Bla fasel\ldots

%% ==============================
\section{Persönlichkeitspsychologie}
%% ==============================
\label{ch:Grundlagen:sec:Abschnitt1}

Zentral für diese Arbeit ist die Geselligkeit eines Menschen.
Wissenschaftlich fällt das Betrachten dieser Persönlichkeitseigenschaft in den Bereich der Persönlichkeitspsychologie,
oft auch Differenzielle Psychologie genannt. 
Dieser Zweig der Pychologie beschäftigt sich mit den Unterschieden zwischen verschiedenen Personen, 
basierend auf Funktionen, Fähigkeiten und Verhalten, deren Ursprünge und der Konsequenzen der selben.
Sie bildet eine Grundlage für praktische Anwendung unter anderem im klinischen oder pädagogischen Kontext.
\par

Historisch begonnen hat das Fachgebiet mit der Intelligenzforschung und Quantifizierung, 
die auch Heute noch ein wichtiger Grundpfeiler der Differenziellen Psychologie ist.
Im Laufe der Zeit kamen weitere Thematiken hinzu wie Temperatmenteigenschaften, Einstellungen und, für die Arbeit relevant, Sozialverhalten.
\par

Bisher konnte sich keine allgemeine, allumfassende Persönlichkeitstheorie durchsetzten, sondern es gibt eine breite Masse an verschiedenen Ansätzen und Menschenbildern, die von verschiedenen Theoretikern unterstützt werden
\hide{ Beispiel Sigmund Freund?}
Im Rahmen dieser Arbeit wird mit dem "Big Five" Persönlichkeitsmodell gearbeitet, das im deutschsprachigen Raum auch manchmal das "Fünf Faktoren Modell" genannt wird.
Es ist, neben dem Myers-Briggs Type Indicator, eines der bekanntesten, meistgenutzten Modelle und wird der Forschung weithin anerkannt.
Das Big Five Modell bietet sich speziell für diese Arbeit an, da es bei diesem Modell sehr einfach ist, nur den Geselligkeits Aspekt und direkt zusammenhängende Aspekte zu betrachten.

\subsection{Big Five Modell}

Nach dem Big Five Modell existieren fünf Hauptdimensionen, die die Grundlage für die Persönlichkeit eines Menschen bilden.
Jeder Mensch lässt sich demnach auf 5 Skalen einordnen, wie sehr die folgendende Eigenschaften bei ihr ausgeprägt sind.

\begin{itemize}
  \item Openness to Experience (Offenheit für Erfahrungen)
  \item Neuroticism (Neurotizismus)
  \item Conscientiousness (Gewissenhaftigkeit)
  \item Agreeableness (Verträglichkeit)
  \item Extraversion
\end{itemize}

Das Big Five Modell verfolgt hierbei einen lexikalischen Ansatz: 
Persönlichkeitsmerkmale schlagen sich in der Sprache der sprechenden Person nieder.
Eine Faktorenanalyse über eine Liste von 18.000 Begriffen ergab die fünf oben stehenden Faktoren.
Diese bleiben über die Lebensspanne stabil und können in den verschiedensten Kulturen beobachtet werden.

\subsection{NEO-PI-R}

Der "NEO - Personality Inventory - Revised" ist ein Persönlichkeitstest, der das Big Five Modell implementiert.
Der NEO-PI-R ist eine 1990 erstmals veröffentlichte überarbeitete Version des NEO-Pi's von 1978 und hat seitdem mehrere Aktualisierungen bekommen. Die aktuellste Version ist von 2010.
Basierend auf 241 Aussagen, zu denen die Probandin mit 5 Abstufungen von "Starke Ablehnung" bis "Starke Zustimmung" ihre Meinung darlegt, legt der Test die 5 Hauptfaktoren plus jeweils 6 Unterfacetten dar.
Die 6 Facetten des für diese Arbeit relevanten Extraversion Faktors sind:

\begin{itemize}
  \item Warmth (Herzlichkeit)
  \item Gregariousness (Geselligkeit)
  \item Assertiveness (Durchsetzungsfähigkeit)
  \item Activity (Aktivität)
  \item Excitement-Seeking (Erlebnishunger)
  \item Positive Emotions (Frohsinn)
\end{itemize}

%% ==============================
\section{Abschnitt 2}
%% ==============================
\label{ch:Grundlagen:sec:Abschnitt2}

Bla fasel\ldots

%% ==============================
\section{Verwandte Arbeiten}
%% ==============================
\label{ch:Grundlagen:sec:RelatedWork}
Hier kommt "`Related Work"' rein.
Eine Literaturrecherche sollte so vollständig wie möglich sein,
relevante Ansätze müssen beschrieben werden und es sollte deutlich 
gemacht werden, wo diese Ansätze Defizite aufweisen oder nicht
anwendbar sind, z.\,B. weil sie von anderen Umgebungen oder 
Voraussetzungen ausgehen.


Bla fasel\ldots

%%% Local Variables: 
%%% mode: latex
%%% TeX-master: "diplarb"
%%% End: 
