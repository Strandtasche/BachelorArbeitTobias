%% Einleitung.tex
%% $Id: einleitung.tex 28 2007-01-18 16:31:32Z bless $
%%

\chapter{Einleitung}
\label{ch:Einleitung}
%% ==============================

Seit jeher sind Wissenschaftler daran interessiert, möglichst präzise, objektive Daten zu sammeln.
Psychologen sind dabei keine Ausnahme.
Besonders im Rahmen der Persönlichkeitspsychologie war es in der Vergangenheit schwer, diese objektiven Daten zu erlangen:
Entweder mussten die Testprobanden von unabhängigen Wissenschaftlern beobachtet werden,
was einerseits viele Ressourcen verbraucht und andererseits, je nach Themengebiet,
einen tiefen Eingriff in die Privatsphäre darstellt. Alternativ waren die Forscher gezwungen,
sich auf durch die Probanden selbst durchgeführten Selbsteinschätzungen zu verlassen,
die sowohl subjektiv sind, als auch als Mühsal empfunden werden können.\par

Mit dem Aufkommen von Smartphones als sensorengespikte Taschencomputer, die in jedem Aspekt des täglichen Lebens Einzug finden,
eröffnen sich ganz neue Möglichkeiten für Psychologen, wie Geoffrey Miller in seinem Paper \textbf{„The Smartphone Psychology Manifesto”} 
darlegt. In dieser Arbeit soll von diesen neuen Möglichkeiten Gebrauch gemacht werden.\par




%% ==============================
\section{Zielsetzung der Arbeit}
%% ==============================
\label{ch:Einleitung:sec:Zielsetzung}

Was ist die Aufgabe der Arbeit?

Bla fasel\ldots

%% ==============================
\section{Gliederung der Arbeit}
%% ==============================
\label{ch:Einleitung:sec:Gliederung}

Was enthalten die weiteren Kapitel?

Bla fasel\ldots

%%% Local Variables: 
%%% mode: latex
%%% TeX-master: "diplarb"
%%% End: 
