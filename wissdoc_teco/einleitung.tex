%% Einleitung.tex
%% $Id: einleitung.tex 28 2007-01-18 16:31:32Z bless $
%%

\chapter{Einleitung}
\label{ch:Einleitung}
%% ==============================

Das Benutzen von Smartphones ist inzwischen ein omnipräsentes Phänomen,
dass nicht mehr nur auf die junge technologiebegeisterte Generation begrenzt ist,
sondern auch die Mitte der Gesellschaft erreicht hat\cite{smartphonealter}.
Smartphones werden benutzt um das Web zu surfen, Nachrichten mit Individuen und Gruppen auszutauschen, Social Media zu konsumieren beziehungsweise zu produzieren und gelegentlich auch um einen Anruf zu tätigen.
Sie haben sich zu einem Grundpfeiler in unserem Kommunikationsverhalten entwickelt.

\par

TODO: Ausschmücken.


%% ==============================
\section{Motivation}
%% ==============================

Seit jeher sind Wissenschaftler daran interessiert, möglichst präzise, objektive Daten zu sammeln.
Psychologen sind dabei keine Ausnahme.
Besonders im Rahmen der Persönlichkeitspsychologie war es in der Vergangenheit schwer, diese objektiven Daten zu erlangen:
Entweder mussten die Testprobanden von unabhängigen Wissenschaftlern beobachtet werden,
was einerseits viele Ressourcen verbraucht und andererseits, je nach Themengebiet,
einen tiefen Eingriff in die Privatsphäre darstellt. Alternativ waren die Forscher gezwungen,
sich auf durch die Probanden selbst durchgeführten Selbsteinschätzungen zu verlassen,
die sowohl subjektiv sind, als auch als Mühsal empfunden werden können.\par

Mit dem Aufkommen von Smartphones als sensorengespikte Taschencomputer, die in jedem Aspekt des täglichen Lebens Einzug finden,
eröffnen sich ganz neue Möglichkeiten für Psychologen, wie Geoffrey Miller in seinem Paper \textbf{„The Smartphone Psychology Manifesto”} \cite{miller2012smartphone}
darlegt. In dieser Arbeit soll von diesen neuen Möglichkeiten Gebrauch gemacht werden.\par




%% ==============================
\section{Zielsetzung der Arbeit}
%% ==============================
\label{ch:Einleitung:sec:Zielsetzung}

Diese Arbeit soll den Zusammenhang zwischen der Nutzung von Social Media, Instant Messaging und der Geselligkeit des Nutzers erforscht werden.
Die Geselligkeit eines Nutzers wird dabei entsprechend des "`Big Five"' Persönlichkeitsmodells als Facette der Extraversion interpretiert und betrachtet.

Dafür wird eine Studie durchgeführt, die sich aus zwei Teilen zusammensetzt:
Dem Feststellen der Geselligkeit der einzelnen Testprobanden und dem Sammeln von Daten bezüglich deren Nutzungsverhalten von Social Media und Instant Messaging. 
Festgestellt wird die Geselligkeit eines Probanden durch den NEO-PI-R Persönlichkeitstest.
Dies ist ein von Paul Costa und Robert McCrae entworfener und in der Persönlichkeitspsychologie breit akzeptierten Fragebogen, der das Big Five Persönlichkeitsmodell nutzt \cite{costa1992neo}.
Das Sammeln von Daten geschieht über eine Android Applikation, die auf dem Smartphone der Teilnehmer installiert wird und dort für zwischen 10 und 20 Tagen Daten sammelt.
Besonderes Augenmerk wird dabei auf die Notifications gelegt, die dem User angezeigt werden, da diese von der Android Nutzerin so konfiguriert werden, dass die für sie relevanten Informationen angezeigt werden.
Zusätzlich dazu werden Daten vom Call Log und Message Log des Android Betriebssystems hinzugezogen.
Vervollständigt werden sollen diese Daten durch Benutzungsdaten, der als relevant klassifizierten Applikationen.

\ignore{todo: Entscheiden ob da die Usagestats rein sollen}

%% ==============================
\section{Gliederung der Arbeit}
%% ==============================
\label{ch:Einleitung:sec:Gliederung}

Im ersten Kapitel der Arbeit ist das Thema sowie die Zielsetzung der Arbeit festgehalten worden, sowie eine kurze Gliederung der restlichen Arbeit.
\par

Im darauf folgenden Kapitel wird zunächst einmal die Grundlage für die weiteren Kapitel gelegt, das heißt
es werden für das Verständnis der Arbeit relevante Begriffe erklärt, 
\par

\ignore{todo}

Im dritten Kapitel, das sich an das Grundlagen Kapitel anschließt wird ein Blick geworfen auf den wissenschaftlichen Hintergrund der Thematik,
 mit der sich diese Arbeit befasst.
Bereits existierende Arbeiten werden begutachtet und bewertet, sodass die Anforderungen an die eigene Arbeit ausgearbeitet werden können und festgestellt werden kann was bei dieser Arbeit anders gemacht werden wird.
\par

Das vierte Kapitel befasst sich als Entwurfskapitel mit der Planung und Vorbereitung der Studie.
Dies beinhaltet sowohl die praktische Herangehensweise an den Persönlichkeitstest als auch das Design beziehungsweise Implementierung der Applikation,
die dann auf den Smartphones der Probanden installiert wird und dort Daten sammelt. 

Das fünfte Kapitel beschreibt die Durchführung der Studie, Probleme die bei dieser Durchführung auftraten und die implementierten Lösungen.

Die Auswertung der während der Studie gewonnenen Daten wird im sechsten Kapitel beschrieben.

Im letzten Kapitel wird eine übergreifende Zusammenfassung der Erkenntnisse, die im Rahmen der Arbeit erhalten wurden, gegeben.
Hinzu kommen Schlussfolgerungen und ein Ausblick auf die Zukunft der Forschung in diesem Feld.


%%% Local Variables: 
%%% mode: latex
%%% TeX-master: "diplarb"
%%% End: 
