%% analyse.tex
%% $Id: analyse.tex 28 2007-01-18 16:31:32Z bless $

\chapter{Analyse}
\label{ch:Analyse}
%% ==============================

Nach der Klarstellung der grundlegenden Begrifflichkeiten im vorhergehenden Kapitel
sollen nun die Anforderungen und Limitationen einer Studie festgestellt werden, die den
Zusammenhang zwischen der Nutzung von Social Media, Instant Messaging und der Geselligkeit
von Nutzern untersuchen soll. 
Dazu werden auch bereits existierende Lösungsansätze betrachtet. 

%% ==============================
\section{Anforderungen}
%% ==============================
\label{ch:Analyse:sec:Anforderungen}

%Anforderungen und Randbedingungen\index{Randbedingungen} \ldots

Für die Untersuchung des Zusammenhanges werden zwei Datenpunkte benötigt.
Auf der einen Seite ein wie auch immer gearteter Wert für die Geselligkeit der Probandin. 
Dem gegenüberstehend wird ein Datensatz der Informationen über das Nutzungsverhalten ebenjener Probandin enthält benötigt.
\par

\subsection{Geselligkeit}

Um die Geselligkeit der Studienteilnehmerinnen festzustellen wird, entsprechend des Big Five Persönlichkeitsmodells, ein von Psychologen entwickelter Selbstbewertungsfragebogen eingesetzt.
Der NEO-Personality Inventory-Revised Fragebogen ist ein in wissen

jadajdajajdajdjajd.

\subsection{Nutzungsverhalten}

Das Nutzungsverhalten der Studienteilnehmerinnen soll mit einer Android Applikation aufgezeichnet werden.
Die Applikation soll mit so wenig Nutzerinteraktion wie möglich auskommen.
Als Aspekt der für das Aufzeichnen des Nutzerverhaltens betrachtet wird werden drei verschiedene Wege aufgezeigt.
\par
Zum ersten die konventionellen Kommunikationsmöglichkeiten eines normalen Mobiltelefons.
Daten zu Anrufen und Kurznachrichten (SMS) lassen sich in Android mit Zugang zum Call Log und zum Message Log auslesen.
Die Berechtigungen dafür können von der Nutzerin bei der Installation erteilt werden und müssen nicht nocheinmal bestätigt werden.
Für das Sammeln dieser Daten muss nicht aktiv das Verhalten der Nutzerin geloggt werden und es muss auch keine Datenbank dazu angelegt werden,
da das Android Betriebssystem dies bereits erledigt.
Es genügt beim finalen Abgeben der Daten auf die Call und Message Logs zuzugreifen und die relevanten Daten auszulesen.
Um ein aussagekräftiges Ergebniss zu erhalten ist die Wahl der relevanten Daten, die untersucht werden, von entscheidender Bedeutung. 
Aus diesem Grund wurden für diese Arbeit verschiedene Ansätze der Wahl der Daten in Betracht gezogen. 
Dabei verspricht der Ansatz aus \cite{chittaranjan2011s} für unsere Zwecke auf Grund seiner breiten Fächerung die besten Resultate.
\par

\subsubsection{Call Log}

\begin{itemize}
    \item Anzahl Anrufe gesamt
    \begin{itemize}
        \item Anzahl Anrufe ausgehend
        \item Anzahl Anrufe eingehend
        \item Anzahl Anrufe verpasst
    \end{itemize}

    \item Anrufdauer gesamt
    \begin{itemize}
        \item Dauer Anrufe ausgehend gesamt
        \item Dauer Anrufe eingehend gesamt
    \end{itemize}

    \item Durschnittliche Dauer gesamt
    \begin{itemize}
        \item Durschnittliche Dauer Anrufe ausgehend
        \item Durschnittliche Dauer Anrufe eingehend
    \end{itemize}

    \item Unique Anrufpartner
    \item Maximale Anzahl Anrufe mit einem Partner

\end{itemize}

\subsubsection{Message Log}

\begin{itemize}
    \item Anzahl SMS gesamt
    \begin{itemize}
        \item Anzahl SMS gesendet
        \item Anzahl SMS empfangen
    \end{itemize}

    \item SMS Zeichenanzahl gesamt
    \begin{itemize}
        \item SMS Zeichenanzahl gesendet gesamt
        \item SMS Zeichenanzahl empfangen gesamt
    \end{itemize}

    \item Durchschnittliche SMS Zeichenanzahl gesamt
    \begin{itemize}
        \item Durchschnittliche SMS Zeichenanzahl gesendet
        \item Durchschnittliche SMS Zeichenanzahl empfangen
    \end{itemize}

\end{itemize}


\subsubsection{Notifications}


\ignore{todo: refereces}
Über den im Grundlagenkapitel bereits vorgestellten NotificationManager kann auf dem Device der Nutzerin eine SQLite Datenbank angelegt werden,
in die alle während dem Studienzeitraum der Nutzerin gezeigten Notifications gespeichert werden.
Die Spalten der SQLite Datenbank sind die Folgenden:
\begin{requirements}
    \req [\_id] fortlaufende Indentifikationsnummer
    \req [notificationEntry] Package der Applikation von der die Notification gepostet wurde
    \req [titleHashed] Gehashter Titel der Notification, Bei Nachrichten oft der Absender
    \req [textLength] Anzahl Zeichen im Text der Notification
    \req [date] Zeitpunkt, an dem die Notification gepostet wurde
\end{requirements}

Im Sinne des Schutzes der Privatssphäre beinhalten diese Einträge nicht exakt den Inhalt der Notification, sondern so anonymisiert, dass die für die Studie relevante Information erhalten bleibt.
So wird der Titel der Notification mit dem SHA-1 Hash\cite{sha1def} gehasht.
SHA-1 ist eine 1995 von US Amerikanischen Nachrichtendienst NSA veröffentlichte cryptographische Hashfunktion\cite{sha1proposal}.
Dies führt dazu, dass für die Auswertung der Studie zwar zwischen zwei verschiedenen Absendern von Nachrichten unterschieden werden kann, diese aber nicht frei zu erkennen sind.
Ebenso speichert die Applikation für die Studie nur die Länge der erhaltenen Nachrichten und nicht deren Inhalt, da dies ein sehr harter Eingriff in die Privatssphäre wäre.
Der Zeitpunkt wird im Datenformat Millisekunden seit Epoch angegeben.



\subsection{Datenschutz}

Die Daten mit denen im Rahmen dieser Studie gearbeitet werden sind sehr sensibler Natur.
Dementsprechend muss sicher gestellt werden, dass sowohl die Anonymität der einzelnen Probandinnen gewahrt bleibt,
als auch dass die Daten nicht den Rahmen der Studie verlassen.
Zusätzlich wird, wie bei allen Studien dieser Art, die am TECO durchgeführt werden,
eine unterschriebene Einwilligungserklärung seitens der Probandin benötigt.
In dieser wird das Einverständnis der Probandin bezüglich der Aufzeichnung, Speicherung und Verarbeitung zugesichert.
\ignore{siehe Anhang, adde anhang?}
\par



%% ==============================
\section{Existierende Lösungsansätze}
%% ==============================
\label{ch:Analyse:sec:RelatedWork}

Hier kommt eine ausführliche Diskussion
von "`Related Work"'.

Bla fasel\ldots

%% ==============================
\section{Weiterer Abschnitt}
%% ==============================
\label{ch:Analyse:sec:Abschnitt}


  \ignore{Bla fasel\ldots hat auch schon \cite{TB20} gesagt und
\cite{TB98,JSAC96,qosr} sollte man mal gelesen haben.
Abbildung~\ref{fig:test} auf S.~\pageref{fig:test} sollte man
sich mal anschauen.}

Blindtext Blindtext Blindtext Blindtext Blindtext Blindtext Blindtext
Blindtext Blindtext Blindtext Blindtext Blindtext Blindtext Blindtext
Blindtext Blindtext Blindtext Blindtext Blindtext Blindtext Blindtext
Blindtext Blindtext Blindtext Blindtext Blindtext Blindtext Blindtext
Blindtext Blindtext Blindtext Blindtext Blindtext Blindtext Blindtext
Blindtext Blindtext Blindtext Blindtext Blindtext Blindtext Blindtext
Blindtext Blindtext Blindtext Blindtext Blindtext Blindtext Blindtext
Blindtext Blindtext Blindtext Blindtext Blindtext Blindtext Blindtext
Blindtext Blindtext Blindtext Blindtext Blindtext Blindtext Blindtext

Blindtext Blindtext Blindtext Blindtext Blindtext Blindtext Blindtext
Blindtext Blindtext Blindtext Blindtext Blindtext Blindtext Blindtext
Blindtext Blindtext Blindtext Blindtext Blindtext Blindtext Blindtext
Blindtext Blindtext Blindtext Blindtext Blindtext Blindtext Blindtext
Blindtext Blindtext Blindtext Blindtext Blindtext Blindtext Blindtext
Blindtext Blindtext Blindtext Blindtext Blindtext Blindtext Blindtext
Blindtext Blindtext Blindtext Blindtext Blindtext Blindtext Blindtext
Blindtext Blindtext Blindtext Blindtext Blindtext Blindtext Blindtext
Blindtext Blindtext Blindtext Blindtext Blindtext Blindtext Blindtext
Blindtext Blindtext Blindtext Blindtext Blindtext Blindtext Blindtext
Blindtext Blindtext Blindtext Blindtext Blindtext Blindtext Blindtext
Blindtext Blindtext Blindtext Blindtext Blindtext Blindtext Blindtext
Blindtext Blindtext Blindtext Blindtext Blindtext Blindtext Blindtext

Blindtext Blindtext Blindtext Blindtext Blindtext Blindtext Blindtext
Blindtext Blindtext Blindtext Blindtext Blindtext Blindtext Blindtext
Blindtext Blindtext Blindtext Blindtext Blindtext Blindtext Blindtext
Blindtext Blindtext Blindtext Blindtext Blindtext Blindtext Blindtext
Blindtext Blindtext Blindtext Blindtext Blindtext Blindtext Blindtext
Blindtext Blindtext Blindtext Blindtext Blindtext Blindtext Blindtext
Blindtext Blindtext Blindtext Blindtext Blindtext Blindtext Blindtext
Blindtext Blindtext Blindtext Blindtext Blindtext Blindtext Blindtext
Blindtext Blindtext Blindtext Blindtext Blindtext Blindtext Blindtext
Blindtext Blindtext Blindtext Blindtext Blindtext Blindtext Blindtext

Blindtext Blindtext Blindtext Blindtext Blindtext Blindtext Blindtext
Blindtext Blindtext Blindtext Blindtext Blindtext Blindtext Blindtext
Blindtext Blindtext Blindtext Blindtext Blindtext Blindtext Blindtext
Blindtext Blindtext Blindtext Blindtext Blindtext Blindtext Blindtext
Blindtext Blindtext Blindtext Blindtext Blindtext Blindtext Blindtext
Blindtext Blindtext Blindtext Blindtext Blindtext Blindtext Blindtext
Blindtext Blindtext Blindtext Blindtext Blindtext Blindtext Blindtext
Blindtext Blindtext Blindtext\index{Blindtext} Blindtext Blindtext Blindtext Blindtext

\begin{figure}[!htbp]
  \centering
  \fbox{\parbox{0.8\textwidth}{
  Abbildungen sollten möglichst als EPS (Encapsulated Postscript) 
  bzw. PDF eingebunden werden.
  Zur Erzeugung sauberer EPS-Dateien empfiehlt sich das Tool \texttt{ps2eps}
  zur Nachbearbeitung von Postscript-Dateien. Mit \texttt{epstopdf} kann
  dann eine PDF-Datei zum Einbinden erzeugt werden.}}
  \caption{Testabbildung}
  \label{fig:test}
\end{figure}

Blindtext Blindtext Blindtext Blindtext Blindtext Blindtext Blindtext
Blindtext Blindtext Blindtext Blindtext Blindtext Blindtext Blindtext
Blindtext Blindtext Blindtext Blindtext Blindtext Blindtext Blindtext
Blindtext Blindtext Blindtext Blindtext Blindtext Blindtext Blindtext
Blindtext Blindtext Blindtext Blindtext Blindtext Blindtext Blindtext
Blindtext Blindtext Blindtext Blindtext Blindtext Blindtext Blindtext
Blindtext Blindtext Blindtext Blindtext Blindtext Blindtext Blindtext

Blindtext Blindtext Blindtext Blindtext Blindtext Blindtext Blindtext
Blindtext Blindtext Blindtext Blindtext Blindtext Blindtext Blindtext
Blindtext Blindtext Blindtext Blindtext Blindtext Blindtext Blindtext

Blindtext Blindtext Blindtext Blindtext Blindtext Blindtext Blindtext
Blindtext Blindtext Blindtext Blindtext Blindtext Blindtext Blindtext
Blindtext Blindtext Blindtext Blindtext Blindtext Blindtext Blindtext
Blindtext Blindtext Blindtext Blindtext Blindtext Blindtext Blindtext
Blindtext Blindtext Blindtext Blindtext Blindtext Blindtext Blindtext
Blindtext Blindtext Blindtext Blindtext Blindtext Blindtext Blindtext

Blindtext Blindtext Blindtext Blindtext Blindtext Blindtext Blindtext
Blindtext Blindtext Blindtext Blindtext Blindtext Blindtext Blindtext
Blindtext Blindtext Blindtext Blindtext Blindtext Blindtext Blindtext
Blindtext Blindtext Blindtext Blindtext Blindtext Blindtext Blindtext
Blindtext Blindtext Blindtext Blindtext Blindtext Blindtext Blindtext
Blindtext Blindtext Blindtext Blindtext Blindtext Blindtext Blindtext
Blindtext Blindtext Blindtext Blindtext Blindtext Blindtext Blindtext
Blindtext Blindtext Blindtext Blindtext Blindtext Blindtext Blindtext
Blindtext Blindtext Blindtext Blindtext Blindtext Blindtext Blindtext
Blindtext Blindtext Blindtext Blindtext Blindtext Blindtext Blindtext
Blindtext Blindtext Blindtext Blindtext Blindtext Blindtext Blindtext
Blindtext Blindtext Blindtext Blindtext Blindtext Blindtext Blindtext
Blindtext Blindtext Blindtext Blindtext Blindtext Blindtext Blindtext
Blindtext Blindtext Blindtext Blindtext Blindtext Blindtext Blindtext

Blindtext Blindtext Blindtext Blindtext Blindtext Blindtext Blindtext
Blindtext Blindtext Blindtext Blindtext Blindtext Blindtext Blindtext
Blindtext Blindtext Blindtext Blindtext Blindtext Blindtext Blindtext
Blindtext Blindtext Blindtext Blindtext Blindtext Blindtext Blindtext
Blindtext Blindtext Blindtext Blindtext Blindtext Blindtext Blindtext
Blindtext Blindtext Blindtext Blindtext Blindtext Blindtext Blindtext
Blindtext Blindtext Blindtext Blindtext Blindtext Blindtext Blindtext
Blindtext Blindtext Blindtext Blindtext Blindtext Blindtext Blindtext
Blindtext Blindtext Blindtext Blindtext Blindtext Blindtext Blindtext
Blindtext Blindtext Blindtext Blindtext Blindtext Blindtext Blindtext
Blindtext Blindtext Blindtext Blindtext Blindtext Blindtext Blindtext
%% ==============================
\section{Zusammenfassung}
%% ==============================
\label{ch:Analyse:sec:zusammenfassung}

Am Ende sollten ggf. die wichtigsten Ergebnisse nochmal in \emph{einem}
kurzen Absatz zusammengefasst werden.

%%% Local Variables: 
%%% mode: latex
%%% TeX-master: "diplarb"
%%% End: 
