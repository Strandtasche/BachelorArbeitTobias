%% zusammenf.tex
%% $Id: zusammenf.tex 4 2005-10-10 20:51:21Z bless $
%%

\chapter*{Kurzfassung}
\label{ch:GermanAbstract}
%% ==============================

%Geselligkeit ist ein großer Bestandteil davon, wie wir mit unserer Umwelt, unseren Mitmenschen interagieren.
Die Geselligkeit einer Person beeinflusst stark die Art und Weise, wie sie mit ihrer Umwelt und Mitmenschen interagiert. 
Mit dem Smartphone als zunehmend wichtiger werdendem Bestandteil unseres Alltags und als omnipräsentes Kommunikationsutensil ist 
die Betrachtung unseres mobilen Kommunikationsverhaltens einfacher denn je. 
\par
In dieser Arbeit wird daher eine Studie in Analyse, Entwurf und Durchführung beschrieben, die die Zusammenhänge zwischen Geselligkeit beziehungsweise Extraversion und Kommunikationsdaten, die auf modernen Smartphones verfügbar sind, erforschen soll. 
Hierfür werden sowohl konventionelle Kommunikationswege wie Anrufe und SMS als auch Kommunikation über Social Media und Messaging Applikationen auf Korrelation mit der Geselligkeit des Nutzers untersucht.
Auf dem Smartphone der Studienteilnehmerinnen wird dazu eine im Rahmen dieser Arbeit entwickelte Android Applikation installiert.
Diese Applikation betrachtet, unter anderem, die eingehenden Notifications als valide Repräsentation relevanter eingehender Nachrichten.
\par
Die durchgeführte Studie führt für 23 Probandinnen die Ergebnisse des 240 Fragen umfassenden NEO-PI-R Fragebogens mit den per Android Applikation über einen Zeitraum von 10 bis 20 Tagen gesammelten Smartphonedaten zusammen.
\par
TODO: ERGEBNISSE BESCHREIBEN! Ausschmücken.
 


%%% Local Variables: 
%%% mode: latex
%%% TeX-master: "diplarb"
%%% End: 
